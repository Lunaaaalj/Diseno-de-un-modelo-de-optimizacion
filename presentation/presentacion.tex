\documentclass[8pt]{beamer}
\usetheme{Berlin}
\usepackage[spanish]{babel}
\usepackage{amsmath}
\usepackage{amssymb}
\usepackage{graphicx}



\title[Modelación de optimización]{Diseño de un modelo de optimización\thanks{\hyperlink{Github}{https://github.com/Lunaaaalj/Diseno-de-un-modelo-de-optimizacion}}}
\author[Ángel, Rubén]{Ángel Luna \and Ángel Rubén}
\institute{Instituto Tecnológico y de Estudios Superiores de Monterrey}
\date{\today}

\begin{document}
\frame{\titlepage}

\begin{frame}
    \frametitle{Problema}
    Un país pequeño se encuentra en guerra y actualmente cuenta únicamente con dos fábricas activas para abastecer al ejército. Las prioridades estratégicas establecidas por el alto mando militar son la producción de armas, alimentos y municiones.

Los recursos disponibles permiten producir como máximo 150 armas, 600 unidades de alimento y 1200 municiones. La fábrica 1 produce por cada hora de operación 3 armas, 2 unidades de alimento y 1 munición, mientras que la fábrica 2 produce 1 arma, 3 unidades de alimento y 2 municiones por hora.

El costo de operación por hora es de 10 000 USD para la fábrica 1 y 7 000 USD para la fábrica 2. Para poder sostener el esfuerzo bélico, se requiere producir al menos 20 armas, 200 unidades de alimento y 150 municiones.

Además, por razones logísticas y estratégicas, se ha determinado que la fábrica 2 debe operar al menos el doble de horas que la fábrica 1.

El objetivo es determinar cuántas horas debe operar cada fábrica de forma que se minimice el costo total de producción, cumpliendo todas las restricciones establecidas.

\end{frame}

\begin{frame}
    \frametitle{Modelación del problema}

\begin{align*}
        \min\quad z &= 10000x_1 + 7000x_2
    \end{align*}
    
    \begin{align*}
        \text{s.a.}\quad &
        \begin{cases}
            3x_1 + x_2 \geq 20 & \text{(armas mínimas)}\\
            3x_1 + x_2 \leq 150 & \text{(armas máximas)}\\[4pt]
            2x_1 + 3x_2 \geq 200 & \text{(alimento mínimo)}\\
            2x_1 + 3x_2 \leq 600 & \text{(alimento máximo)}\\[4pt]
            x_1 + 2x_2 \geq 150 & \text{(municiones mínimas)}\\
            x_1 + 2x_2 \leq 1200 & \text{(municiones máximas)}\\[4pt]
            x_2 \geq 2x_1 & \text{(restricción operativa)}\\
            x_1, x_2 \geq 0 & \text{(no negatividad)}
        \end{cases}
    \end{align*}

\end{frame}

\begin{frame}
    \frametitle{Graficación}
    \begin{figure}
        \centering
        \includegraphics[width = 0.8\textwidth]{plot.png}
    \end{figure}
\end{frame}

\begin{frame}[fragile]
    \frametitle{Solución}
    \begin{verbatim}
model = pyo.ConcreteModel()

model.x1 = pyo.Var(within = pyo.NonNegativeIntegers)
model.x2 = pyo.Var(within = pyo.NonNegativeIntegers)

model.obj = pyo.Objective(expr = 10000 * model.x1 + 7000 * model.x2, 
    sense = pyo.minimize)

model.con1 = pyo.Constraint(expr = 3 * model.x1 + model.x2 >= 20)
model.con2 = pyo.Constraint(expr = 3 * model.x1 + model.x2 <= 150)
model.con3 = pyo.Constraint(expr = 2 * model.x1 + 3 * model.x2 >= 200)
model.con4 = pyo.Constraint(expr = 2 * model.x1 + 3 * model.x2 <= 600)
model.con5 = pyo.Constraint(expr = model.x1 + 2 * model.x2 >= 150)
model.con6 = pyo.Constraint(expr = model.x1 + 2 * model.x2 <= 1200)
model.con7 = pyo.Constraint(expr = model.x2 >= 2 * model.x1)

opt = pyo.SolverFactory("glpk")
results = opt.solve(model)
model.display()
    \end{verbatim}
\end{frame}

\begin{frame}[fragile]
    \frametitle{Resultados}
    \begin{verbatim}
        Variables:
    x1 : Size=1, Index=None
        Key  : Lower : Value : Upper : Fixed : Stale : Domain
        None :     0 :   0.0 :  None : False : False : NonNegativeIntegers
    x2 : Size=1, Index=None
        Key  : Lower : Value : Upper : Fixed : Stale : Domain
        None :     0 :  75.0 :  None : False : False : NonNegativeIntegers

  Objectives:
    obj : Size=1, Index=None, Active=True
        Key  : Active : Value
        None :   True : 525000.0
    \end{verbatim}
\end{frame}

\begin{frame}
    \frametitle{Conclusiones y Recomendaciones}
    
    \textbf{Recomendaciones estratégicas:}
    \begin{itemize}
        \item La fábrica 1 debe permanecer inactiva ($x_1 = 0$ horas)
        \item La fábrica 2 debe operar 75 horas ($x_2 = 75$ horas)
        \item Costo total óptimo de operación: \$525,000 USD
        \item Esta configuración cumple todos los requisitos mínimos de producción al menor costo posible
    \end{itemize}
    
    \vspace{0.3cm}
    
    \textbf{Análisis de sensibilidad:}
    \begin{itemize}
        \item Si aumentan los recursos disponibles, podrían ajustarse las restricciones máximas
        \item Si cambian los requisitos mínimos de producción, podría requerirse activar la fábrica 1
        \item Cambios en los costos operativos podrían modificar la solución óptima
        \item La restricción operativa ($x_2 \geq 2x_1$) actualmente se cumple sin afectar la optimalidad
    \end{itemize}
    
    \vspace{0.3cm}
    
    \textbf{Gracias por su confianza.} Quedamos a su disposición para ajustes adicionales o análisis futuros según cambien las condiciones operativas.
\end{frame}

\end{document}